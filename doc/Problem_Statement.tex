\documentclass[12pt]{article}
\usepackage{amsmath,latexsym,amsfonts,amsthm,graphicx,subcaption}
\usepackage[margin=1in]{geometry} \usepackage{tikz,caption}
\usetikzlibrary{backgrounds} \usepackage{fancyhdr}
\usepackage{multicol} \usepackage{caption} \usepackage{empheq}
%\setlist{parsep=0pt,listparindent=\parindent}
\usepackage{listings} 

\definecolor{light-gray}{gray}{0.95}

\lstset{%numbers=right, 
                %numberstyle=\tiny, 
                breaklines=true,
                backgroundcolor=\color{light-gray},
                %numbersep=5pt,
                xleftmargin=.25in,
                xrightmargin=0in} 



\newcommand{\ep}{\epsilon} \newcommand{\noi}{\noindent}
\newcommand{\sol}{\subsubsection*{Solution}}
\newcommand{\R}{\mathbb{R}} \newcommand{\inti}{\int_{-\infty}^\infty}
\newcommand{\wto}{\rightharpoonup} \newcommand{\bd}{\partial \Omega}
\newcommand{\pb}[1]{\subsection*{Problem #1}}
\newcommand{\ang}[2]{\langle #1 , #2 \rangle} \newcommand{\<}{\langle}
\newcommand{\sa}{\sigma\text{-algebra}}
\newcommand{\pr}{\partial}
\begin{document} 
\thispagestyle{fancy} 
\chead{CSE380: Final Project Proposal} 
\lhead{Gopal Yalla} 
\rhead{11/11/2016}

\section{$v^2$-$f$ Model}

\subsection{Equations} 
\begin{align}
	\pr_t u_i + u_j d_j u_i &= \frac{-1}{\rho} \pr_i \left( P +
	\frac{2}{3} \rho k \right) + \pr_j \left[ (\nu + \nu_T)(\pr_j u_i + \pr_i
	u_j)]\right] \\
	\nu_T &= C_\mu \overline{v^2} T \\ 
	\pr_t k + u_j d_j k &= \mathcal{P}- \ep + \pr_j \left( (\nu +\nu_T)d_j
	k\right)\\
	\pr_t \ep + u_j d_j \ep &= \frac{C_{\ep 1}\mathcal{P} - C_{\ep 2}
	\ep}{T} + \pr_j \left( (\nu + \frac{\nu_T}{\sigma_\ep}) \pr_j \ep \right)\\
	\pr_t \overline{v^2} + u_j d_j \overline{v^2} + \ep
	\frac{\overline{v^2}}{k} &= kf + \pr_k [ \nu_T \pr_k \overline{v^2}] + v
	\nabla^2 \overline{v^2} \\
	L^2 \nabla^2 f - f &= -c_2 \frac{\mathcal{P}}{k} + \frac{c_1}{T} \left(
	\frac{\overline{v^2}}{k} - \frac{2}{3} \right) 
	\label{eq:v2f}
\end{align}

\subsection{Constants} 

\begin{multicols}{2}
\begin{enumerate}
	\item $C_\mu = 0.09$
	\item $\sigma_\ep = 1.3$
	\item $c_2 = 0.3$	
	\item $C_L = 0.23$
\end{enumerate}
\columnbreak 

\begin{itemize}
	\item[5.] $c_1 = 0.4$
	\item[6.] $C_{\ep 1} = 1.4[1 + 0.045 (k/v^2)^{1/2}]$
	\item[7.] $C_{\ep 2} = 1.92$  
	\item[8.] $C_\eta = 70$
\end{itemize}
\end{multicols}

\subsection{Terms and Unknowns}

\textbf{Unknowns:}
\begin{itemize}
	\item $u_i = i^{th}$ component of the mean velocity. 
	\item $\nu_T = $ eddy viscosity.
	\item $k =$ turbulent kinetic energy. 
	\item $\ep =$ energy dissipation (to eddy viscosity.)
	\item $\overline{v^2} =$ Velocity scale to capture anisotropy near wall
	\item $f =$ ``redistribution term.''
\end{itemize}

\noi \textbf{Other Terms:} 
\begin{itemize}
	\item $\mathcal{P} = $ rate at which mean flow is converted to turbulent
		fluctuation. \newline
		$= - \overline{u_i u_j} \pr_j u_i$ \newline
		$= 2 \nu_T |S|^2$ for incompressible flow. 

	\item $S = $ mean rate of strain tensor \newline
		$= 1/2 ( \pr_i u_j + \pr_j u_i)$ \newline
		$= 1/2 d_y u$ for steady state problem
	\item $T = $ turbulent length scale. \newline
		$= k/\ep$ 
	\item $L = $ length scale \newline
		$= \max \left\{ C_L \frac{k^{3/2}}{\ep}, C_\eta (
		\frac{v^3}{\ep} )^{1/4} \right\}$
		
\end{itemize}

\subsection{Steady State Equations for Fully Developed Flow}

\begin{align}
	(\nu + \nu_T) \frac{\pr^2 u}{\pr y^2} + \frac{\pr u}{\pr y}
	\frac{\pr \nu_T}{\pr y} - \frac{1}{\rho} \frac{\pr P }{\pr x} &= 0 \\
	\nu_T - C_\mu \overline{v^2} T &=0 \\
	\mathcal{P} - \ep + (\nu+ \nu_T) \frac{\pr^2 k }{\pr y^2} +
	\frac{\pr k}{\pr y} \frac{\pr \nu_T}{\pr y} &=0\\
	\frac{C_{\ep 1 } \mathcal{P} - C_{\ep 2 } \ep}{T} + \left(\nu +
	\frac{\nu_T}{ \sigma_\ep}
	\right) \frac{\pr^2 \ep}{\pr y^2 } + \frac{1}{\sigma_\ep}
	\frac{\pr \ep}{\pr y} \frac{\pr \nu_T}{\pr y} &= 0 \\
	kf + (\nu + \nu_T) \frac{\pr^2 \overline{v^2}}{\pr y^2} +
	\frac{\pr \overline{v^2}}{\pr y} \frac{\pr \nu_T}{\pr y} -
	\frac{\ep \overline{v^2}}{k} &= 0 \\
	L^2 \frac{d^2 f }{dy^2} - f + c_2 \frac{\mathcal{P}}{k} -
	\frac{c_1}{T} \left( \frac{\overline{v^2}}{k} - \frac{2}{3}\right) &=0
	\label{eq:ssv2f}
\end{align}


\section{Computational Problem} 

We consider the flow through a rectangular duct of height $h=2 \delta$. The mean
flow is predominantly in the axial direction with the mean velocity varying in
the cross-stream direction. The bottom and top walls are at $y = 0$ and $y =
2 \delta$ with the mid plane being $y = \delta$. We confine our attention to the
fully developed region in which staistics no longer vary with $x$. Hence, the
fully developed channel flow being considered is statistically stationary and
statistically one-dimensional, with velocity statistics depending only on $y$.
Moreover, the flow is statistically symmetric about the mid-plane $y = \delta$. 


\subsection{Initial / Boundary Conditions} 

\textbf{Boundary Conditions}:

\begin{enumerate}
	\item $ u = 0$ at $y = 0$. 
	\item $ k = 0$ at $y = 0$. 
	\item $ \overline{v^2} = 0$ at $y = 0$. 
	\item $ \ep = ?$ at $y = 0$. 
	\item $ f = 0$ at $ y = 0$.
	\item $u_y = 0$ at $y = 0$.
	\item $k_y = 0$ at $y = 0$. 
	\item $\overline{v^2}_y = ?$ at $y = 0$
	\item $\ep_y = ?$ at $y = 0 $. 
\end{enumerate}


\noi \textbf{Initial Conditions (for time marching)}: 
\begin{enumerate}
	\item $u(0,y) = ?$
	\item $k(0,y) = ?$ 
	\item $\ep(0,y) = ?$ 
	\item $\overline{v^2}(0,y) = ?$
	\item $f(0,y) = $none 
\end{enumerate}

\subsection{Other terms to define} 

\begin{enumerate}
	\item $\frac{\pr P}{\pr x} = ?$
	\item $\delta = 1$
	\item $\Delta t = ?$  (small $\rightarrow$ large)
	\item $\Delta y = ?$ 
	\item $\rho = ?$
	\item $\nu = ?$
\end{enumerate}



\end{document} 

